%%%%%%%%%%%%%%%%%%%%%%%%%%%%%%%%%%%%%%%%
% Fancyslides Presentation
% LaTeX Template
% Version 1.0 (30/6/13)
%
% This template has been downloaded from:
% http://www.LaTeXTemplates.com
%
% The Fancyslides class was created by:
% Paweł Łupkowski (pawel.lupkowski@gmail.com)
%
% License:
% CC BY-NC-SA 3.0 (http://creativecommons.org/licenses/by-nc-sa/3.0/)
%
%%%%%%%%%%%%%%%%%%%%%%%%%%%%%%%%%%%%%%%%%

%----------------------------------------------------------------------------------------
%	PACKAGES AND OTHER DOCUMENT CONFIGURATIONS
%----------------------------------------------------------------------------------------

\documentclass{fancyslides}

\usepackage[utf8]{inputenc} % Allows the usage of non-english characters
\usepackage{times} % Use the Times font
\usepackage{booktabs} % Allows the use of \toprule, \midrule and \bottomrule in tables
\usepackage{comment}
\graphicspath{{images/}} % Location of the slide background and figure files

% Beamer options - do not change
\usetheme{default} 
\setbeamertemplate{navigation symbols}{} % Disable the slide navigation buttons on the bottom of each slide
\setbeamercolor{structure}{fg=\yourowntexcol} % Define the color of titles and fixed text elements (e.g. bullet points)
\setbeamercolor{normal text}{fg=\yourowntexcol} % Define the color of text in the presentation

%------------------------------------------------
% COLORS
% The following colors are predefined in this class: white, black, gray, blue, green and orange

% Define your own color as follows:
%\definecolor{pink}{rgb}{156,0,151}

\newcommand{\structureopacity}{0.75} % Opacity (transparency) for the structure elements (boxes and circles)

\newcommand{\strcolor}{blue} % Set the color of structure elements (boxes and circles)
\newcommand{\yourowntexcol}{white} % Set the text color

%----------------------------------------------------------------------------------------
%	TITLE SLIDE
%----------------------------------------------------------------------------------------

\newcommand{\titlephrase}{1.2.0 Reunión de la Comunidad Local de Software Libre y Código Abierto de La Rioja - Manos a la obra} % Presentation title
\newcommand{\name}{Emmanuel Arias} % Presenter's name
\newcommand{\affil}{eamanu.com} % Presenter's institution
\newcommand{\email}{emmanuelarias30@gmail.com//eamanu@eamanu.com} % Presenter's email address


\begin{document}
	
	\fbckg{blank.jpg} % Slide background image
	\begin{frame}
	\centering
	\includegraphics[scale=0.25]{gnu.png}
	\end{frame}

\startingslide % This command inserts the title slide as the first slide

%----------------------------------------------------------------------------------------
%	PRESENTATION SLIDES
%----------------------------------------------------------------------------------------

\fbckg{1.jpg} % Slide background image
\begin{frame}
\framedsl{Estemos en contacto} % Text in this environment is printed in a circle and will be made large and uppercase - if you need to fit more text in you can reduce the font size within the \pointedsl{} bracket as usual, e.g. \pointedsl{\large smaller main point}
\end{frame}

\fbckg{2.jpg} % Slide background image
\begin{frame}
		\itemized{ 
	Mandar un email a emmanuelarias30@gmail.com
	
	\textit{Subject: [SL La Rioja] Contacto}
	
    \textit{To: emmanuelarias30@gmail.com}
	
	Hola, \\Quisiera unirme a la comunidad de software libre de la Rioja.\\ Saludos!
}
\end{frame}

\fbckg{2.jpg} % Slide background image
\begin{frame}
	\itemized{ 
		\item Channel IRC (Freenode): \#LaRiojaSoft
		\item Chat Gitter (https://gitter.im)
		\item Via mail
		\item Slack
	}
\end{frame}

\fbckg{1.jpg} % Slide background image
\begin{frame}
	\framedsl{Primeras Tareas} % Text in this environment is printed in a circle and will be made large and uppercase - if you need to fit more text in you can reduce the font size within the \pointedsl{} bracket as usual, e.g. \pointedsl{\large smaller main point}
\end{frame}

\fbckg{2.jpg} % Slide background image
\begin{frame}	
	\itemized{ 
		\item Poner Nombre al grupo. Votación. ¿Logo?
		\item Elegir repositorio y crear cuenta. Github? Gitlab? Otro?
		\item Dónde vamos a alojar nuestros sistemas?
		\item Crear Página
		\item Formalizar el grupo
		\item Crear firma. GnuPG (https://gnupg.org/)
	}
\end{frame}

\fbckg{1.jpg} % Slide background image
\begin{frame}
	\framedsl{Luego...} % Text in this environment is printed in a circle and will be made large and uppercase - if you need to fit more text in you can reduce the font size within the \pointedsl{} bracket as usual, e.g. \pointedsl{\large smaller main point}
\end{frame}

\fbckg{2.jpg} % Slide background image
\begin{frame}
	\itemized{ 
		\item Elegir proyectos para Contribuir. Crear un bug tracker?
		\item Lluvia de ideas de proyectos para generar.
		\item Segunda Reunión. Empaquetado de Debian.
		\item Estudiemos el Kernel?
	}
\end{frame}

\fbckg{1.jpg} % Slide background image
\begin{frame}
\thankyou % Inserts a thank you slide
\end{frame}

\fbckg{blank.jpg} % Slide background image
\begin{frame}	
	\itemized{ % This environment simply prints a series of bullet points
        https://github.com/eamanu/Talks/blob/master/\\
        2019/ComunidadSoftLibre/Septiembre2019/\\
        PrimeraRunion/actividades.tex
    }
\end{frame}

\fbckg{blank.jpg} % Slide background image
\begin{frame}	
	\itemized{ % This environment simply prints a series of bullet points
emmanuelarias30@gmail.com
	}
\end{frame}

%------------------------------------------------

%----------------------------------------------------------------------------------------

\end{document}