%%%%%%%%%%%%%%%%%%%%%%%%%%%%%%%%%%%%%%%%%
% Fancyslides Presentation
% LaTeX Template
% Version 1.0 (30/6/13)
%
% This template has been downloaded from:
% http://www.LaTeXTemplates.com
%
% The Fancyslides class was created by:
% Paweł Łupkowski (pawel.lupkowski@gmail.com)
%
% License:
% CC BY-NC-SA 3.0 (http://creativecommons.org/licenses/by-nc-sa/3.0/)
%
%%%%%%%%%%%%%%%%%%%%%%%%%%%%%%%%%%%%%%%%%

%----------------------------------------------------------------------------------------
%	PACKAGES AND OTHER DOCUMENT CONFIGURATIONS
%----------------------------------------------------------------------------------------

\documentclass{fancyslides}

\usepackage[utf8]{inputenc} % Allows the usage of non-english characters
\usepackage{times} % Use the Times font
\usepackage{booktabs} % Allows the use of \toprule, \midrule and \bottomrule in tables
\usepackage{comment}
\graphicspath{{images/}} % Location of the slide background and figure files

% Beamer options - do not change
\usetheme{default} 
\setbeamertemplate{navigation symbols}{} % Disable the slide navigation buttons on the bottom of each slide
\setbeamercolor{structure}{fg=\yourowntexcol} % Define the color of titles and fixed text elements (e.g. bullet points)
\setbeamercolor{normal text}{fg=\yourowntexcol} % Define the color of text in the presentation

%------------------------------------------------
% COLORS
% The following colors are predefined in this class: white, black, gray, blue, green and orange

% Define your own color as follows:
%\definecolor{pink}{rgb}{156,0,151}

\newcommand{\structureopacity}{0.75} % Opacity (transparency) for the structure elements (boxes and circles)

\newcommand{\strcolor}{blue} % Set the color of structure elements (boxes and circles)
\newcommand{\yourowntexcol}{white} % Set the text color

%----------------------------------------------------------------------------------------
%	TITLE SLIDE
%----------------------------------------------------------------------------------------

\newcommand{\titlephrase}{EL SOFTWARE LIBRE: EL QUÉ, EL CÓMO Y EL POR QUÉ} % Presentation title
\newcommand{\name}{Emmanuel Arias} % Presenter's name
\newcommand{\affil}{Yaerobi.com} % Presenter's institution
\newcommand{\email}{eamanu@eamanu.com} % Presenter's email address

\begin{document}

\startingslide % This command inserts the title slide as the first slide

%----------------------------------------------------------------------------------------
%	PRESENTATION SLIDES
%----------------------------------------------------------------------------------------

\fbckg{1.jpg} % Slide background image
\begin{frame}
\framedsl{¿Qué es el Software Libre?} % Text in this environment is printed in a circle and will be made large and uppercase - if you need to fit more text in you can reduce the font size within the \pointedsl{} bracket as usual, e.g. \pointedsl{\large smaller main point}
\end{frame}
% Foto de Richard
\fbckg{richard.jpeg}
\begin{frame}
\end{frame}
% Foto de Richard
\fbckg{richard2.jpeg}
\begin{frame}
\end{frame}

% -----------------------------------------------
% Las 4 libertades del software libre
\fbckg{gnu.png} 
\begin{frame}
\framedsl{4 libertades básicas del software} % Text in this environment will be made large, uppercase and will wrap multiple lines
\end{frame}

\fbckg{2.jpg} % Slide background image
\begin{frame}
\itemized{ % This environment simply prints a series of bullet points
\item Libertad 0: libertad de \textbf{usar} el programa, con cualquier propósito (uso). 
}
\end{frame}

\fbckg{2.jpg} % Slide background image
\begin{frame}
	\itemized{ % This environment simply prints a series of bullet points
		\item Libertad 1: libertad de \textbf{estudiar} cómo funciona el programa y modificarlo, adaptándolo a las propias necesidades (estudio). 
	}
\end{frame}

\fbckg{2.jpg} % Slide background image
\begin{frame}
	\itemized{ % This environment simply prints a series of bullet points
		\item Libertad 2: la libertad de \textbf{distribuir} copias del programa, con lo cual se puede ayudar a otros usuarios (distribución). 
	}
\end{frame}

\fbckg{2.jpg} % Slide background image
\begin{frame}
	\itemized{ % This environment simply prints a series of bullet points
		\item Libertad 3: la libertad de \textbf{mejorar} el programa y hacer públicas esas mejoras a los demás, de modo que toda la comunidad se beneficie (mejora). 
	}
\end{frame}

%------------------------------------------------
\begin{comment}
\fbckg{2.jpg} % Slide background image
\begin{frame}
\framedsl{\pitem{First point} \pitem{Second point} \fitem{Third point}} % Text in \pitem commands will be printed one after another on separate slides until all are displayed
\end{frame}
\end{comment}
%------------------------------------------------

%------------------------------------------------
% Quien usa Software libre?
\fbckg{blank.jpg} % Slide background image
\begin{frame}
	\framedsl{Basta de tonterías, ¿quién va a fabricar Software Libre?}
\end{frame}

\fbckg{tux.jpg} % Slide background image
\begin{frame}
	\framedsl{Linux}
\end{frame}

\fbckg{linus.jpg} % Slide background image
\begin{frame}
\end{frame}

\fbckg{google.jpg} % Slide background image
\begin{frame}
\end{frame}

\fbckg{opensource.png} % Slide background image
\begin{frame}
\end{frame}

\fbckg{blank.jpg} % Slide background image
\begin{frame}
	\centering
	\includegraphics[scale=0.3]{eric.jpg}
\end{frame}

%End el qué
%------------------------------------------------

%------------------------------------------------
% Como?
\fbckg{1.jpg} % Slide background image
\begin{frame}
	\framedsl{¿Cómo puedo asegurarme que no me robarán mi creación?} 
\end{frame}

\fbckg{1.jpg} % Slide background image
\begin{frame}
	\framedsl{¿Cómo puedo armar mi negocio sobre software libre?} 
\end{frame}

\fbckg{1.jpg} % Slide background image
\begin{frame}
	\framedsl{¿Cómo puedo asegurarme que no me robarán mi creación?} 
\end{frame}

%------------------------------------------------

\fbckg{2.jpg} % Slide background image
\begin{frame}
\misc{ % Anything can be placed inside the \misc{} command
\Huge
Numbered list:
\begin{enumerate}
\centering	
\item First item
\item Second item
\item Third item
\end{enumerate}
}
\end{frame}

%------------------------------------------------

\fbckg{2.jpg} % Slide background image
\begin{frame}
\misc{ % Anything can be placed inside the \misc{} command
Tables can be included with the \texttt{\textbackslash misc\{\}} command:
\begin{table}[h]
\begin{tabular}{l l l}
\toprule
\textbf{Treatments} & \textbf{Response 1} & \textbf{Response 2}\\
\midrule
Treatment 1 & 0.0003262 & 0.562 \\
Treatment 2 & 0.0015681 & 0.910 \\
Treatment 3 & 0.0009271 & 0.296 \\
\bottomrule
\end{tabular}
\end{table}
}
\end{frame}

%------------------------------------------------

\fbckg{2.jpg} % Slide background image
\begin{frame}
\misc{ % Anything can be placed inside the \misc{} command
Figures can also be included with the \texttt{\textbackslash misc\{\}} command:
\begin{figure}[h]
\includegraphics[width=0.4\linewidth]{placeholder}
\end{figure}
}
\end{frame}

%------------------------------------------------

\fbckg{1.jpg} % Slide background image
\begin{frame}
\thankyou % Inserts a thank you slide
\end{frame}

%------------------------------------------------

\fbckg{blank} % A blank background can be used instead of an image
\begin{frame}
\sources{ % An environment for giving credit for slide backgrounds, images will need to be scaled down if there are more than two
\includegraphics[scale=0.048]{1.jpg} \ flickr/lovelornpoets\\
\includegraphics[scale=0.2]{2.jpg} \ flickr/apsmuseum
}
\end{frame}

%----------------------------------------------------------------------------------------

\end{document}