%%%%%%%%%%%%%%%%%%%%%%%%%%%%%%%%%%%%%%%%
% Fancyslides Presentation
% LaTeX Template
% Version 1.0 (30/6/13)
%
% This template has been downloaded from:
% http://www.LaTeXTemplates.com
%
% The Fancyslides class was created by:
% Paweł Łupkowski (pawel.lupkowski@gmail.com)
%
% License:
% CC BY-NC-SA 3.0 (http://creativecommons.org/licenses/by-nc-sa/3.0/)
%
%%%%%%%%%%%%%%%%%%%%%%%%%%%%%%%%%%%%%%%%%

%----------------------------------------------------------------------------------------
%	PACKAGES AND OTHER DOCUMENT CONFIGURATIONS
%----------------------------------------------------------------------------------------

\documentclass{fancyslides}

\usepackage[utf8]{inputenc} % Allows the usage of non-english characters
\usepackage{times} % Use the Times font
\usepackage{booktabs} % Allows the use of \toprule, \midrule and \bottomrule in tables
\usepackage{comment}
\graphicspath{{images/}} % Location of the slide background and figure files

% Beamer options - do not change
\usetheme{default} 
\setbeamertemplate{navigation symbols}{} % Disable the slide navigation buttons on the bottom of each slide
\setbeamercolor{structure}{fg=\yourowntexcol} % Define the color of titles and fixed text elements (e.g. bullet points)
\setbeamercolor{normal text}{fg=\yourowntexcol} % Define the color of text in the presentation

%------------------------------------------------
% COLORS
% The following colors are predefined in this class: white, black, gray, blue, green and orange

% Define your own color as follows:
%\definecolor{pink}{rgb}{156,0,151}

\newcommand{\structureopacity}{0.75} % Opacity (transparency) for the structure elements (boxes and circles)

\newcommand{\strcolor}{blue} % Set the color of structure elements (boxes and circles)
\newcommand{\yourowntexcol}{white} % Set the text color

%----------------------------------------------------------------------------------------
%	TITLE SLIDE
%----------------------------------------------------------------------------------------

\newcommand{\titlephrase}{Comunidad de Software Libre en La Rioja} % Presentation title
\newcommand{\name}{Emmanuel Arias} % Presenter's name
\newcommand{\affil}{eamanu.com} % Presenter's institution
\newcommand{\email}{emmanuelarias30@gmail.com//eamanu@eamanu.com} % Presenter's email address


\begin{document}
	
	\fbckg{blank.jpg} % Slide background image
	\begin{frame}
	\centering
	\includegraphics[scale=0.25]{gnu.png}
	\end{frame}

\startingslide % This command inserts the title slide as the first slide

%----------------------------------------------------------------------------------------
%	PRESENTATION SLIDES
%----------------------------------------------------------------------------------------

\fbckg{1.jpg} % Slide background image
\begin{frame}
\framedsl{¿Qué es el Software Libre?} % Text in this environment is printed in a circle and will be made large and uppercase - if you need to fit more text in you can reduce the font size within the \pointedsl{} bracket as usual, e.g. \pointedsl{\large smaller main point}
\end{frame}
% Foto de Richard
\fbckg{richard.jpeg}
\begin{frame}
\end{frame}

% -----------------------------------------------
% Las 4 libertades del software libre
\fbckg{gnu.png} 
\begin{frame}
	\framedsl{4 libertades básicas del software} % Text in this environment will be made large, uppercase and will wrap multiple lines
\end{frame}

\fbckg{2.jpg} % Slide background image
\begin{frame}
	\itemized{ % This environment simply prints a series of bullet points
		\item Libertad 0: libertad de \textbf{usar} el programa, con cualquier propósito (uso). 
	}
\end{frame}

\fbckg{2.jpg} % Slide background image
\begin{frame}
	\itemized{ % This environment simply prints a series of bullet points
		\item Libertad 1: libertad de \textbf{estudiar} cómo funciona el programa y modificarlo, adaptándolo a las propias necesidades (estudio). 
	}
\end{frame}

\fbckg{2.jpg} % Slide background image
\begin{frame}
	\itemized{ % This environment simply prints a series of bullet points
		\item Libertad 2: la libertad de \textbf{distribuir} copias del programa, con lo cual se puede ayudar a otros usuarios (distribución). 
	}
\end{frame}

\fbckg{2.jpg} % Slide background image
\begin{frame}
	\itemized{ % This environment simply prints a series of bullet points
		\item Libertad 3: la libertad de \textbf{mejorar} el programa y hacer públicas esas mejoras a los demás, de modo que toda la comunidad se beneficie (mejora). 
	}
\end{frame}

\fbckg{1.jpg} % Slide background image
\begin{frame}
	% This environment simply prints a series of bullet points
	\framedsl{El rol de la Universidad Publica}
\end{frame}

\fbckg{2.jpg} % Slide background image
\begin{frame}
	\itemized{ % This environment simply prints a series of bullet points
		\textit{RMS: El software libre contribuye al conocimiento humano, al contrario del software privativo. Por tanto, las universidades deben impulsar el software libre por el bien del progreso del saber humano, como así también alentar a los científicos y estudiantes a publicar sus trabajos}
	}
\end{frame}

\fbckg{2.jpg} % Slide background image
\begin{frame}
	\itemized{ % This environment simply prints a series of bullet points
		\textit{RMS: Lamentablemente, muchos administradores de universidades adoptan una actitud egoísta con respecto al software (y a la ciencia); consideran los programas como una posible fuente de ingresos, no como oportunidades para contribuir al conocimiento humano.}
	}
\end{frame}

\fbckg{blank.jpg} % Slide background image
\begin{frame}
	% This environment simply prints a series of bullet points
\begin{figure}
	\centering
	\includegraphics[width=1\linewidth]{images/excel}
	\caption{}
	\label{fig:excel}
\end{figure}

\end{frame}

\fbckg{blank.jpg} % Slide background image
\begin{frame}
\begin{figure}
	\centering
	\includegraphics[width=1\linewidth]{images/msproject}
	\caption{}
	\label{fig:msproject}
\end{figure}
\end{frame}

\fbckg{blank.jpg} % Slide background image
\begin{frame}
	% This environment simply prints a series of bullet points
\begin{figure}
	\centering
	\includegraphics[width=1\linewidth]{images/excel2}
	\caption{}
	\label{fig:excel2}
\end{figure}

\end{frame}

\fbckg{blank.jpg} % Slide background image
\begin{frame}
	% This environment simply prints a series of bullet points
	\begin{figure}
		\centering
		\includegraphics[width=1\linewidth]{images/excel3}
		\caption{}
		\label{fig:excel3}
	\end{figure}
	
\end{frame}

\fbckg{blank.jpg} % Slide background image
\begin{frame}
	% This environment simply prints a series of bullet points
\begin{figure}
	\centering
	\includegraphics[width=1\linewidth]{images/excel4}
	\caption{}
	\label{fig:excel4}
\end{figure}
\end{frame}

\fbckg{blank.jpg}
\begin{frame}
	\begin{figure}
		\centering
		\includegraphics[width=1\linewidth]{images/precio_ms_office}
		\caption{}
		\label{fig:preciomsoffice}
	\end{figure}
\end{frame}

\fbckg{blank.jpg}
\begin{frame}
	\begin{figure}
		\centering
		\includegraphics[width=1\linewidth]{images/precio_ms_project}
		\caption{}
		\label{fig:preciomsproject}
	\end{figure}
	
\end{frame}

\fbckg{blank.jpg}
\begin{frame}
	\begin{figure}
		\centering
		\includegraphics[width=1\linewidth]{images/adobe}
		\caption{}
		\label{fig:adobe}
	\end{figure}
\end{frame}

\fbckg{1.jpg} % Slide background image
\begin{frame}
% This environment simply prints a series of bullet points
\framedsl{Vamos a crear una comunidad de Software Libre y Código Abierto.}
\end{frame}

\begin{frame}
	\framedsl{Objetivos}
\end{frame}

\fbckg{2.jpg} % Slide background image
\begin{frame}
	\itemized{ % This environment simply prints a series of bullet points
		\item Contribuir a proyectos de software libre o código abierto (Debian, Python, Kernel, etc..)
		\item Generar proyectos para la provincia/sociedad/región/país. 
		\item Acceder a becas y ayudas para estudiantes y profesionales (Google Summer of Code, Debian Bursary, etc.)
	}
\end{frame}
\fbckg{2.jpg} % Slide background image
\begin{frame}
	\itemized{ % This environment simply prints a series of bullet points
		\item Ganar experiencia.
		\item Estudiar y aprender.
		\item Permitir que los estudiantes avanzados generen Trabajos de Tesis dentro del Grupo.
		\item Alumnos avanzados puedan realizar prácticas profesionales.
		\item Presentar charlas en congresos nacionales (e internacional?)
		\item Reuniones cada 1 1/2 mes. Video calls. 
	}
\end{frame}

\fbckg{1.jpg} % Slide background image
\begin{frame}
\thankyou % Inserts a thank you slide
\end{frame}

\fbckg{blank.jpg} % Slide background image
\begin{frame}	
	\itemized{ % This environment simply prints a series of bullet points
        https://github.com/eamanu/Talks/blob/master/\\
        2019/ComunidadSoftLibre/Septiembre2019/\\
        PrimeraRunion/objetivos.tex
    }
\end{frame}

%------------------------------------------------

%----------------------------------------------------------------------------------------

\end{document}