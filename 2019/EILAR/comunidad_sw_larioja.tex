\documentclass[12pt]{article}
\usepackage[utf8]{inputenc}
\usepackage[spanish]{babel}

\usepackage[left=2.5cm, right=2cm, top=2.5cm, bottom=2cm]{geometry}

\usepackage{helvet}
\renewcommand{\familydefault}{\sfdefault}
\usepackage{graphicx}
\usepackage{float}

% Title Page
\title{Comunidad de Software Libre en La Rioja}
\author{Arias Emmanuel}
\date{2 de septiembre del 2019}

\begin{document}
\maketitle
\newpage

\begin{abstract}
	La Universidad tiene que ser la principal promotora del uso de Software Libre en sus estudiantes, ya que esta institución
	debe velar por las libertades y el conocimiento de ellos. Por lo tanto, el autor busca generar desde la 
	Universidad, la cultura de uso de software libre y de código abierto en los estudiantes del área de sistemas. Se tiene el objetivo de 
	dar a conocer los principales beneficios que se pueden obtener gracias a este movimiento, tales como becas de viaje, becas
	para estudiantes como "Google Summer of Code", y acceso a importantes empleos a nivel mundial. La primera reunión
	para dar a conocer esta idea se realizará el día 5 de septiembre del año en curso. Para la fecha en el que se
	realizará el Encuentro Informático de La Rioja 2019 se tendrán los primeros resultados de este grupo y serán mostrados en el
	evento.
\end{abstract}

\section{Software libre}
La definición de software libre presenta los criterios para decidir si un software en particular califica como libre. Tal como se indica en \cite{freesw}, el ''open source'' es una filosofía muy diferente al SL y se encuentra basado en diferentes valores. 

Como  ya es sabido el SL, se basa en cuatro libertades:
\begin{itemize}
	\item \textbf{Libertad 0}: Es la libertad de ejecutar un programa como se desee, y para cualquier propósito.
	\item \textbf{Libertad 1}: Es la libertad de estudiar cómo funciona un programa, y realizar modificaciones. Para ello es necesario poder acceder al código fuente. 
	\item \textbf{Libertad 2}: Es la libertad de redistribuir copias, con esto, se puede ayudar a las demás personas. 
	\item \textbf{Libertad 3}: Es la libertad de distribuir copias de versiones modificadas del software original. Con esto se le brinda a la comunidad la oportunidad de beneficiarse por los cambios realizados en el software. 
\end{itemize}

Si un software cumple con estas libertades, se puede indicar que este software es ''libre'', en caso contrario es clasificado como ''non-free''.

''Free software'' no significa ''software no comercial''. Un SL puede ser de uso comercial, nada lo impide \cite{freesw}. Por ejemplo, Red Hat es una empresa multinacional que provee SL a empresas. Red Hat crea, mantiene y contribuye a una gran cantidad de proyectos de SL, además adquirió software propietario y ha liberado el código. 

En inglés, surge el problema de que Free Software se confunde con software gratuito, ya que ''free'' tiene el doble significado de ''libre'' y ''gratis''. Esto no sucede en el español, ya que existe una palabra diferente para el concepto de libre y gratis. Pero vale aclarar que SL no es igual a gratuito. 


\section{Open Source}
El movimiento del software libre nace en 1983 de la mano de Richard Stallman, el cual lanza en ese año el \textit{GNU Project}. Luego, en 1985 Stallman crea la \textit{Free Software Foundation}. Tal como lo explica Stallman en \cite{richard-open}, SL no es lo mismo que Open Source. En 1998 una parte de la comunidad del SL se separa del movimiento, y crea lo que denominaron ''open source''. Este término fue adoptado para evitar un posible mal entendimiento de la palabra SL, pero tiene puntos de vistas diferentes al movimiento.  

Si bien, los dos términos describen casi la misma categoría de software, el open source se refiere a una metodología de desarrollo, se preocupa sobre como hacer mejor el software (solo en el sentido práctico). Mientras que el SL es un movimiento social \cite{richard-open}, un compromiso ético que respeta las libertades de los usuarios de software. 

Todo SL clasifica como open source, pero en contraposición, no todo open source clasifica como SL, ya que existen muchas licencias que van en contra de las libertades. 

\section{Comunidad de Sofware Libre en La Rioja}

\subsection{Objetivos}
\begin{itemize}
	\item Desarrollo de la industria de software en La Rioja desde la Universidad Nacional de La Rioja.
	\item Introducción de los estudiantes del área informática al movimiento del software libre.
	\item Fortalecimiento por parte de los estudiantes y graduados de UNLAR en el desarrollo de software.
	\item Divulgación del movimiento del Software Libre.
\end{itemize}

\subsection{Beneficios de la creación de un Grupo de estudio/instituto dedicado al SL}
\begin{itemize}
	\item Visibilidad a nivel nacional y sobre todo mundial. Si se logra realizar
	contribuciones ''importantes'' en el movimiento de SL es posible captar visibilidad
	de otras organizaciones y fundaciones dedicadas al SL. Esto significa
	que se podrá acceder a financiaciones y empresas reclutadoras.

	\item Práctica profesional de alumnos interesados en el desarrollo de software
	mediante la participación de proyectos FOSS.

	\item Fortalecimiento de alumnos, graduados y docentes en el desarrollo de 
	software mediante la participación en proyectos existentes.

	\item Generación de proyectos con licencia open source.

	\item Prestación de servicios de consultoría a empresas y entidades
	gubernamentales.

	\item Accesos a becas que promuevan el uso de software libre. Por ejemplo, Becas de
	viaje completo a la conferencias de Debian, Google Summer of Code, etc.
\end{itemize}

\textit{Para la realización del Encuentro Informático La Rioja 2019 se tendrán resultado
sobre la comunidad.}

\end{document}